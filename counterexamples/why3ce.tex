\documentclass[a4paper,twoside]{article}

% fonts
\usepackage[utf8]{inputenc}
\usepackage[T1]{fontenc}
\usepackage{mathptmx}
\usepackage[scaled=0.8]{beramono}

% geometry
\usepackage[a4paper,centering]{geometry}
\setlength{\marginparwidth}{2cm}

% Inria research report
\usepackage{RR}

% extra packages

\usepackage{hyperref}
\usepackage{xspace}

\usepackage{amsmath}
\usepackage{amsfonts}
\usepackage{amsthm}

\usepackage[color=yellow!45!white,linecolor=black,textsize=footnotesize]{todonotes}

\usepackage{why3lang}

% placement of figures
\renewcommand\floatpagefraction{0.99}
\renewcommand{\textfraction}{0.01}
\renewcommand{\topfraction}{0.99}
\renewcommand{\bottomfraction}{0.99}
\setcounter{topnumber}{4}
% \setcounter{bottomnumber}{4}
% \setcounter{totalnumber}{8}


\newcommand{\ie}{\textit{i.e.\/}\xspace}
\newcommand{\eg}{\textit{e.g.\/}\xspace}


\RRNo{XXXX}    % TODO
% \RTNo{XXXX}
%%
%% date de publication du rapport
\RRdate{xxx 2024}
%%
%% Cas d'une version deux
%% \RRversion{2}
%% date de publication de la version 2
%% \RRdater{November 2008}
%%
\RRauthor{% les auteurs
  Claude Marché\thanks[toccata]{Université Paris-Saclay, CNRS, ENS Paris-Saclay, Inria,
    LMF, 91190, Gif-sur-Yvette, France}
}
%% Ceci apparait sur chaque page paire.
\authorhead{Marché}
%% titre francais long
\RRtitle{Format des contre-exemples\thanks{Ces recherches ont été partiellement financées par Décysif TODO....}}
%% English title
\RRetitle{Counterexamples formats\thanks{This work has been partially supported by Décysif... TODO}}
%%
\titlehead{Counterexamples Formats}
%%
% \RRnote{This is a note}
% \RRnote{This is a second note}
%%
\RRresume{A FAIRE}

\RRabstract{TO DO}

%%
\RRmotcle{Vérification déductive, Solveurs SMT, contre-exemples}
%
\RRkeyword{Deductive Verification, SMT solvers, counterexamples}
%%
\RRprojet{Toccata}  % cas d'un seul projet
% \RRprojets{Apics and Op\'era and Marelle}
%%
%% \URLorraine % pour ceux qui sont \`a l'est
%% \URRennes  % pour ceux qui sont \`a l'ouest
%% \URRhoneAlpes % pour ceux qui sont dans les montagnes
%% \URRocq % pour ceux qui sont au centre de la France
%% \URFuturs % pour ceux qui sont dans le virtuel
%% \URSophia % pour ceux qui sont au Sud.
%%
%% \RCBordeaux % centre de recherche Bordeaux - Sud Ouest
%% \RCLille % centre de recherche Lille Nord Europe
%% \RCParis % Paris Rocquencourt
\RCSaclay % Saclay \^Ile de France
%% \RCGrenoble % Grenoble - Rh\^one-Alpes
%% \RCNancy % Nancy - Grand Est
%% \RCRennes % Rennes - Bretagne Atlantique
%% \RCSophia % Sophia Antipolis M\'editerran\'ee


% \maketitle

\newtheorem{definition}{Definition}[section]
\newtheorem{example}[definition]{Example}
\newtheorem{lemma}[definition]{Lemma}
\newtheorem{corollary}[definition]{Corollary}
\newtheorem{theorem}[definition]{Theorem}


\begin{document}
\sloppy\hbadness=9999
%%
\makeRR   % cas d'un rapport de recherche
% \makeRT % cas d'un rapport technique.
%% a partir d'ici, chacun fait comme il le souhaite


\tableofcontents
\clearpage

\listoffigures

\clearpage

\section{Introduction}

What is deductive verification, how SMT solvers are used, how counterexamples are produced

Overview of the document.

\section{Counterexamples pour les buts logiques}


Pour un but logique:
\begin{lstlisting}
constant g1 : t    (* constante globale non-interpretee *)

constant g2 : t    (* constante globale non-interpretee *)

function f y1 ... yn : t   (* function logique non-interpretee *)

predicate p z1 .. zm     (* predicat non-interprete *)

goal g : forall x1 ... xk. F
\end{lstlisting}

un contre-example doit proposer:
\begin{itemize}
\item une valeur pour chaque constante non-interprétée
\item une valeur fonctionnelle pour chaque fonction non-interpretée
\item une valeur fonctionnelle pour chaque prédicat non-interpreté
\item une valeur pour les variables universellement quantifiées en tete du but
\end{itemize}
Les valeurs des objets du contexte doivent etre données aussi pour les objets
qui viennent de modules importés.


La validation d'un tel contre-exemple doit procéder en construisant la tache ``instanciée''
\begin{lstlisting}
constant g1 : t = ... (* valeur proposee pour g1 *)

constant g2 : t = ...  (* valeur proposee pour g2 *)

function f y1 ... yn : t = ... (* definition proposee pour f *)

predicate p z1 .. zm = ...  (* definition proposee pour p *)

constant x1 = ... (* valeur proposee pour x1 *)

...

constant xk = ... (* valeur proposee pour xk *)

goal g : not F
\end{lstlisting}
si cette tache est prouvée, alors le CE est validé

Note: si le modele proposé par le solveur est incomplet, on peut générer une
tache similaire, où les valeurs inconnues restent des déclarations non
interprétées. Si la tache instanciée est prouvée, le CE est validé aussi.

Note: un ``one-liner'' contre-exemple serait la restriction des valeurs aux
seuls variables liées $z_1,..,x_k$.

Si la tache instanciée n'est pas prouvée, on peut tenter de tester la meme tache
mais avec le but sans negation. Si cette seconde tache est prouvée, le CE est
forcement invalide et doit etre rejeté. Si aucune des taches avec F ou not F
n'est prouvée, on peut dire a l'utilisateur que le cCE n'a pas pu etre validé,
et lui proposer le CE sans garantie.


\begin{example}
Exemple ajouté comme test de référence pour Decysif:

\lstinputlisting{../../benchmarks/why3_ce/tests_for_log/log-logic.mlw}
\end{example}

\section{Contrexemple pour des VCs d'un programme: Cas d'un code sans conditionnelle, sans boucle et sans appel de fonction}

Ce cas peut se résumer à la forme

\begin{lstlisting}
<contexte global, avec objets non-interpretes>

let f x1 .. xk : t
  requires { P }
  ensures { Q }
= ... (* assignments, local vars with let .. in *)
  assert { A }
  ...;
  e
\end{lstlisting}
Les parametres peuvent etre des references mutables, de meme que les variables
locales peuvent etre mutables aussi.

pour la VC qui concerne l'assertion $A$, un contre-exemple doit proposer
\begin{itemize}
\item Des valeurs pour chaque objet logique non-interprété du contexte
\item des valeurs pour les parametres $x_1,..x_k$, valeurs ``initiales'' si ce
  sont des references mutables.
\end{itemize}

La validation d'un CE proposé consiste alors à executer le code depuis l'entrée
de la fonction $f$ avec les valeurs initiales de $x_1,..,x_k$. Lorsque l'execution atteint l'assertion, avec un environnement d'execution $E$ qui donne des valeurs courantes pour les $x_1...,x_k$ ainsi que les variables locales, la formule $A$ doit etre testée avec une tache de la forme
\begin{lstlisting}
(* definitions proposees pour les objets logiques globaux *)

constant x1 = ... (* valeur courante pour x1 calculee par l'execution *)

...

constant xk = ... (* valeur courante pour xk calculee par l'execution *)


constant l1 = ... (* valeurs courantes pour les variables locales, calculees par l'execution *)

goal g : not A
\end{lstlisting}
si la tache est prouvée, le contre-exemple est validée.

Le ``one-liner'' CE consiste en les valeurs courantes de $x_1,..x_k,l1,...$



\begin{example}
Exemple ajouté comme test de référence pour Decysif:

\lstinputlisting{../../benchmarks/why3_ce/tests_for_log/log-simple.mlw}
\end{example}

\subsection{cas de plusieurs assertions}

Lorsque l'execution atteint une assertion A' qui n'est pas encore l'assertion
cible de la VC, il faut verifier que la tache instanciée sans negation sur A'
est prouvée. Si oui on peut avancer. Dans le cas contraire, on peut tester la
tache avec negation, si celle-ci est prouvée on peut rapporter une assertion
fausse sur A'. Sinon on peut continuer l'execution mais au final indiquer que le
contre-exemple n'a pas eté entierement validé.


\subsection{cas de la post-condition}

Le meme procédé permet de tester un CE pour la preuve de la post-condition. le
modele du solveur doit alors indiquer la valeur v renvoyée par la fonction. La
tache instanciée doit alors definir result comme egal a cette valeur.

\section{Cas des conditionnelles}

L'execution a partir du debut de la fonction evalue les conditions et par
consequent decide quelle branche de la conditionnelle est prise. Le CE doit
alors, en plus des valeurs précitées, indiquer quel branche de chaque
conditionnelle a été prise avant d'atteindre l'assertion testée.

\subsection{Resumé : execution sous oracle}

Les procédés decrits précédemment utilisent donc une forme d'execution sous oracle,
qui demarre en entre de fonction, defini un environnement d'execution initial
avec les definitions logiques pour les objects logiques non-interpretés, et des
valeurs initiales pour les parametres. L'execution peut alors se derouler, en
ajoutant les variavbles locales au fur et a mesure, en enregistrant les branches
choisies dans les conditionnelles, et pour chaque assertion A traversée, si A ou
not A etait prouvée, et ce dans quel contexte courant (d'ou sera extrait le
``one-liner'').

Note technique pour cette execution: pour chaque assertion une tache de preuve
doit etre construite pour appeler un prouveur. On remarque que chacune de ces
taches possede un contexte commun formé des valeurs proposées pour les objets
logiques globaux. Pour une efficacité de cette execution, il est souhaitable que
ce repfixe de tache soit calculé une fois pour toute et reutilisé. Ceci n'est
pas le cas dans l'implementation actuelle du RAC dans Why3: see discussion in
\url{https://gitlab.inria.fr/why3/why3/-/issues/829}.

Note additionelle: la tache contexte en question pourrait aussi accumuler en
hypotheses la connaissance des assertions prouvées, car cela peut aider les
prouveurs ensuite. Mais attention ces assertions sont dans un environnement
d'execution anterieur. Cela voudrait donc dire accumuler des definitions pour
les valeurs successives des variables, ce que l'on ne veut pas forcement faire.


\section{Cas des appels de fonctions}


Dans le cas d'un appel de fonction, on souhaite utiliser un ``pas de geant'':
evaluer la precondition, a traiter comme une assertion, puis obtenir des valeurs
modifiées pour les variables ecrites par la fonction appelee, ainsi que la
valeur de retour, et dans un tel nouvel envionnement, evaluer la post-condition
comme une assertion.

\section{Cas des boucles}

Pour une boucle, l'execution doit commencer pour tester si l'invariant est
initialement vrai, comme pour une assertion. Ensuite, on doit consulter l'oracle
pour obtenir des valeurs pour les variables modifiées par la boucle. Celles-ci
doivent donc etre present dans le modele proposés par le solveur. Il s'agit la
des valeurs des variables modifiées avant une prochaine itération (eventuelle) de
la boucle. Avec ces valeurs, on teste la condition de boucle.

Si la condition est vraie, on mémorise, comme pour une conditionnelle, que l'on
va effectivement entrer dans la boucle. On continue l'execution en executant le
corps de la boucle. Si l'execution atteint alors une assertion fausse, on a
finit, mais si on atteint la fin du corps de boucle sans assertion fausse, alors
on enregistre que l'execution est ``stuck'' car cela n'a pas de sens de
continuer.

Si la condition de boucle est fausse, on poursuit l'execution avec ce qui suit
le corps de boucle.

La trace d'execution doit contenir l'information indiquant si on est entré dans
le corps de boucle ou bien si on a sauté a la fin, de facon similaire a une
conditionnelle.


\section{Cas des exceptions}

En cas de rencontre d'un raise, il faut considérer que l'evaluation produit
l'exception associée, et propager jusqu'a rencontrer un try..with..

Noter que dans le cas ou une exception est levee dans un corps de boucle, par
exemple avec break, alors on peut se retrouver a executer ce qui suit la boucle
alors meme que l'on était entré dans la boucle. La bonne facon d'implementer
cela est avec les regles données dans le rapport~\cite{becker21rr}

Les exceptions peuvent etre levées par un appel de fonction, auquel cas on doit
tester la post-condition exceptionnelle associée, et continuer l'execution comme
si l'on avait rencontré un raise.


\bibliographystyle{plainurl}
\bibliography{abbrevs,demons,demons2,demons3,team,crossrefs}
%\bibliography{generated,extra}

%\clearpage


\end{document}

% Local Variables:
% mode: latex
% TeX-master: t
% TeX-PDF-mode: t
% mode: flyspell
% ispell-local-dictionary: "british"
% fill-column: 80
% End:
