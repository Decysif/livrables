\documentclass[a4paper,11pt]{article}

\usepackage[utf8]{inputenc}
\usepackage[T1]{fontenc}
\usepackage{mathptmx}

\usepackage[a4paper,text={160mm,220mm},centering]{geometry}

\usepackage{hyperref}
\usepackage{listings}
\usepackage{graphicx}
\usepackage[table]{xcolor}

\lstset{basicstyle={\ttfamily}}

\usepackage{todonotes}

\begin{document}
\sloppy\hbadness=9999

\thispagestyle{empty}

\unitlength=1mm
\begin{picture}(140,30)
\put(-10,30){\includegraphics[height=12mm]{../images/TIS-logo.png}}
\put(-10,11){\includegraphics[height=22mm]{../images/adacore.png}}
\put(-6,0){\includegraphics[height=12mm]{../images/logo_ocamlpro.png}}
\put(125,30){\includegraphics[height=10mm]{../images/Universite_Paris_Saclay_logo.png}}
\put(127,12){\includegraphics[height=16mm]{../images/cnrs.png}}
\put(123,0){\includegraphics[height=12mm]{../images/logo-inria-reduced.png}}
\end{picture}

\vfill

\begin{center}

{  \Huge\bfseries
  Projet Décysif --- Livrable 3.1 }

\bigskip

{ \LARGE\bfseries Memory Models for Pointer Programs: a State of the
  Art from the Point of View of Décysif Partners }


\vfill

\large January 2025

\vfill

\large
% Yannick Moy (AdaCore),
% Guillaume Cluzel (TrustInSoft),
% Matteo Manighetti (Inria \& Université Paris-Saclay),
Claude Marché (Inria \& Université Paris-Saclay)


\end{center}

\vfill

\noindent\begin{picture}(140,30)
\put(0,0){\includegraphics[width=0.3\textwidth]{../images/Logo_Bpifrance.png}}
\put(70,0){\includegraphics[width=0.3\textwidth]{../images/LOGO_RIDF_2019_COULEUR.png}}
\put(145,0){\includegraphics[width=0.1\textwidth]{../images/Logo-France-2030-rouge-bleu.png}}
\end{picture}

\noindent Le projet Décysif est financé par la Région Île-de-France et par le Gouvernement
Français dans le cadre du Plan France 2030

\clearpage

\tableofcontents
\clearpage

% \listoffigures
% \clearpage

\section{Introduction}

The term \emph{Deductive Verification} denotes the methodology for verifying
that a computer programs is conforming to a specification of its intended
behaviour, with the use of techniques based of automated theorem proving. In
this context, the specification is written in some formal mathematical language,
and the conformity of the code with respect to the specification is
automatically reduced into a set of mathematical formulas than must proved
valid. When the computer programs considered are expressed in a programming
language making use of pointers to access their memory heap, the interpretation
of programs into mathematical formulas must rely on a logical description of
such pointers and memory heap: the so-called \emph{memory model}.

When reasoning mathematically on pointer programs, the suitability of the memory
model is crucial. One major challenge is the potential \emph{aliasing} of memory
contents: when using pointers, the same memory cell can be denoted by
multiple ``names'', or ``access paths''. Modification of the memory content
through one of these names should be taken into account when accessing the same
memory cell via another name. More generally, \emph{blocks} of memory cells can be read
or written through such access paths, and these blocks may \emph{overlap}. The property of
non-overlapping for memory blocks is called their \emph{separation}. Separation
assumptions play an important role when reasoning about pointer
programs~\cite{bornat00mpc}.

A memory model cannot be described uniformly among different programming
languages. On the contrary, a specific memory model must be defined for each
programming language considered. It is even possible to consider several memory
models for different supported subsets of a given programming language, as in
the Wp plug-in of the Frama-C environment for analysis of C
code~\cite{blanchard2024wp}.

The Décysif project gathers partners involved in the design of
verification environments for several programming languages, involving
different approaches for memory access. They develop multiple tools and each of them
contains a different approach for dealing with aliasing and separation: the tool
Why3 for WhyML programs, the SPARK environment for Ada code, the J3
analyzer for C and C++ code, and the Creusot tool for Rust code.

Generally speaking, there are numerous technical challenges related to the
design of memory models. This is not the purpose of this document to review
them. Instead we recommend to the interested reader the reading of a survey on
that purpose by Leavens \emph{et. al.}~\cite{leavens07}.  The purpose of this
document is focused to summarise the different memory models implemented in the
Decysif partners tools, fosuing on the important design choices, and briefly
compare them with each other and with other existing approaches in the
international community of deductive verification. From such a summary, we
derive and expose some future work we want to perform in the context of the
Décysif project.

TODO: Overview of the document: Section~\ref{sec:why3}, Section~\ref{sec:spark},
Section~\ref{sec:j3}, Section~\ref{sec:creusot}, Section~\ref{sec:related},
Section~\ref{sec:future}.

\section{The memory model of Why3}
\label{sec:why3}

The Why3
environment~\cite{filliatre13esop,bobot14sttt,blazy19fmtea,paskevich20isola} is
designed and developed by the Inria partner of Décysif. It is not an environment
for a particular mainstream programming language such as C or Java. Instead, it
is designed as a \emph{generic} environment for deductive verification. This
genericity comes into different aspects. First, Why3 proposes its own
programming language, called WhyML, dedicated to deductive verification, in the
sense that it is designed to ease the generation of verification conditions,
making the proof of these formulas as automatic as possible. This relies on a
second generic aspect of Why3: its ability to communicate with a large set of
provers to discharge the verification conditions~\cite{boogie11why3}. A third
generic aspect of Why3 is its suitability to be used as an intermediate language
for deductive verification: as a matter of fact, WhyML is used as an
intermediate language by the other tools mentioned below: Spark, J3 and Creusot.

In the WhyML language, mutability of memory is supported, but to control the
difficulties coming from aliasing and separation, mutation of memory is
controlled by a dedicated powerful static typing system. This typing system is
expressed on top of important notions of \emph{ownership} and
\emph{regions}~\cite{gondelman16reg}. Thanks to this typing system, the
potential issues with aliasing and memory separation are handled \emph{during}
the generation of VCs, so that the generated formulas do not need to take
aliasing anymore into account. Roughly speaking, the generated VCs are as simple
as if they were coming from a purely functional programming language, with no
memory mutability at all. One must understand in particular that this approach
differs fundamentally with other approaches well-known in the litterature, such
as \emph{separation logic} or \emph{dynamic frames}, where non-aliasing and
separation properties are still present in the logic formulas to be proved
valid, which makes the work of provers significantly more difficult. The
powerful region-based mechanism of Why3 comes at a cost though: the set of
programs that one is writing in WhyML must pass the type checking, which means
the programmer must think carefully about how the side effects in their program
must be expressed.

The research work around Why3 is still active nowadays, e.g. regarding
the design of better constructs for writing specifications, the
improvement of the ratio of proof automation, or the generation of
counterexamples when proofs
fail~\cite{dailler18jlamp,becker21fide}. In the context of Décysif,
the efforts are planned on two directions: the design of an
alternative to WhyML, and the improvement of counterexamples
generation to better suit the need of frontends of the partners. These
are details in Section~\ref{sec:future} below.

\section{The memory model of Spark}
\label{sec:spark}

  \todo{Premier jet par Claude, rédaction à reprendre par AdaCore, Claire ?}

Complete absence of pointers at first, then partial support of pointers, aliased
controlled using a borrow checker~\cite{dross20cav,jaloyan20icfem}.

How the translation to Why3 works.

Common current issues. E.g. with specification, with proof automation, with
counterexamples.

\section{The memory model of J3}
\label{sec:j3}

Aims at supporting low-level C code, with arbitrary pointer casts. Inspired by
CompCert V2~\cite{leroy12rr}. Herms thesis~\cite{herms13phd}.

How the translation to Why3 works. Bitvectors everywhere~\cite{fumex16nfm}.

Not yet supported but future work: static analysis
of separation similar to the old plugin
Jessie~\cite{hubert2008these,hubert07hav}.


\paragraph{positioning}

Frama-C/WP~\cite{blanchard2024wp}

Frama-C/Jessie , separation analysis~\cite{hubert2008these,hubert07hav}

C code, separation logic, VST, Appel Bertot, Verifast


\section{The memory model of Creusot}
\label{sec:creusot}

  \todo{Premier jet par Claude, rédaction à reprendre par l'équipe Creusot, Jacques-Henri ?}

Relies on Rust's borrow checker to propose a translation to Why3. mutable
borrows are translated to pairs of current value and future final value
(prophecy).\cite{denis23phd,denis22icfem,matsushita22pldi}.

\paragraph{positioning}

Prusti~\cite{astrauskas19oopsla}, aeneas~\cite{ho22icfp}


\section{Comparison with Tools for Other Languages}
\label{sec:related}

Java, JML, data groups~\cite{breunesse03ftfjp}, universes~\cite{Dietl05jot}

Spec\# ~\cite{BarnettLS04}, dynamic frames, Dafny

CFML~\cite{chargueraud08icfp}, gospel~\cite{gospelfm19}, Cameleer~\cite{DBLP:conf/cav/PereiraR20}


\section{Future Work within Décysif}
\label{sec:future}



\begin{description}
\item[Flexible Abstraction Barriers]

  New intermediate language in Why3, called Coma, continuation-based. On going thesis of
  Paul Patault.

\item[From Counterexamples to Exploits]

  Better support for counterexamples, needs to makes memory models
  more concrete, more ``executable-like'', to obtain directly test
  case in frontends to be replayed as exploits.

\item[Future Work in SPARK]
  \todo{Rédacteur Souhaité: AdaCore, Claire ?}

  Exploits ?

  Add prophecies in the Spec lang of SPARK?

\item[J3]

  Separation analysis for its memory model

  Executability for exploit

\item[Extensions of Creusot]

  Support for concurrency in Creusot, Support for ghost aliasing, on-going
  thesis of Arnaud Golfouse.

\item[CEs in Creusot]

\end{description}

\bibliographystyle{plainurl}
%\bibliography{abbrevs,demons,demons2,demons3,team,crossrefs}
\bibliography{generated,extra}

%\clearpage


\end{document}

% Local Variables:
% mode: latex
% TeX-master: t
% TeX-PDF-mode: t
% mode: flyspell
% ispell-local-dictionary: "british"
% fill-column: 80
% End:
