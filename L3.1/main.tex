\documentclass[a4paper,11pt]{article}

\usepackage[utf8]{inputenc}
\usepackage[T1]{fontenc}
\usepackage{mathptmx}

\usepackage[a4paper,text={160mm,220mm},centering]{geometry}

\usepackage{hyperref}
\usepackage{listings}
\usepackage{graphicx}
\usepackage[table]{xcolor}

\lstset{basicstyle={\ttfamily}}

\usepackage{todonotes}

\begin{document}
\sloppy\hbadness=9999

\thispagestyle{empty}

\unitlength=1mm
\begin{picture}(140,30)
\put(-10,30){\includegraphics[height=12mm]{../images/TIS-logo.png}}
\put(-10,11){\includegraphics[height=22mm]{../images/adacore.png}}
\put(-6,0){\includegraphics[height=12mm]{../images/logo_ocamlpro.png}}
\put(125,30){\includegraphics[height=10mm]{../images/Universite_Paris_Saclay_logo.png}}
\put(127,12){\includegraphics[height=16mm]{../images/cnrs.png}}
\put(123,0){\includegraphics[height=12mm]{../images/logo-inria-reduced.png}}
\end{picture}

\vfill

\begin{center}

{  \Huge\bfseries
  Projet Décysif --- Livrable 3.1 }

\bigskip

{ \LARGE\bfseries Memory Models for Pointer Programs: a State of the
  Art from the Point of View of Décysif Partners }


\vfill

\large January 2025

\vfill

\large
Claude Marché (Inria \& Université Paris-Saclay),
Guillaume Cluzel (TrustInSoft),
Claire Dross (AdaCore),
Jean-Christophe Filliâtre (CNRS \& Université Paris-Saclay),
Jacques-Henri Jourdan (CNRS \& Université Paris-Saclay),
Andrei Paskevich (Université Paris-Saclay),


\end{center}

\vfill

\noindent\begin{picture}(140,30)
\put(0,0){\includegraphics[width=0.3\textwidth]{../images/Logo_Bpifrance.png}}
\put(70,0){\includegraphics[width=0.3\textwidth]{../images/LOGO_RIDF_2019_COULEUR.png}}
\put(145,0){\includegraphics[width=0.1\textwidth]{../images/Logo-France-2030-rouge-bleu.png}}
\end{picture}

\noindent Le projet Décysif est financé par la Région Île-de-France et par le Gouvernement
Français dans le cadre du Plan France 2030

\clearpage

\tableofcontents
\clearpage

% \listoffigures
% \clearpage

\section{Introduction}

The term \emph{Deductive Verification} denotes the methodology for verifying
that a computer programs is conforming to a specification of its intended
behavior, with the use of techniques based of automated theorem proving. In
this context, the specification is written in some formal mathematical language,
and the conformity of the code with respect to the specification is
automatically reduced into a set of mathematical formulas than must proved
valid. When the computer programs considered are expressed in a programming
language making use of pointers to access their memory heap, the interpretation
of programs into mathematical formulas must rely on a \emph{logical description} of
such pointers and of the memory heap: the so-called \emph{memory model}.

When reasoning mathematically on pointer programs, the suitability of the memory
model is crucial. One major challenge is the potential \emph{aliasing} of memory
contents: when using pointers, the same memory cell can be denoted by multiple
``names'', or ``access paths''. Modification of the memory content through one
of these names should be taken into account when accessing the same memory cell
via another name. More generally, \emph{blocks} of memory cells can be read or
written through such access paths, and these blocks may \emph{overlap}. The
property of non-overlapping for memory blocks is called their
\emph{separation}. Separation assumptions play an important role when reasoning
about pointer programs, an early reference publication on that subject is due to
Bornat in 2000~\cite{bornat00mpc}.  Since then, many different approaches for
dealing with memory aliasing and separation have been proposed, including the
landmark proposal \emph{separation logic}~\cite{reynolds02lics}. We discuss them
more in Section~\ref{sec:related}.

A memory model cannot be described uniformly among different programming
languages. On the contrary, a specific memory model must be defined for each
programming language considered. It is even possible to consider several memory
models for different supported subsets of a given programming language, as in
the WP plug-in of the Frama-C environment for analysis of C
code~\cite{blanchard2024wp}.
%
The Décysif project gathers partners involved in the design of
verification environments for several programming languages, involving
different approaches for memory access. They develop multiple tools and each of them
contains a different approach for dealing with aliasing and separation: the tool
Why3 for WhyML programs, the SPARK environment for Ada code, the J3
analyzer for C and C++ code, and the Creusot tool for Rust code.

Generally speaking, there are numerous technical challenges related to the
design of memory models. This is not the purpose of this document to review
them. Instead we recommend to the interested reader the reading of a survey on
that purpose by Leavens \emph{et. al.}~\cite{leavens07} published in 2007.  The
purpose of this document is to summarize the different memory models
implemented in the tools of the Décysif partners, focusing on the important design
choices, and briefly compare them with each other and with other existing
approaches in the international community of deductive verification. From such a
summary, we derive and expose some future work we want to perform in the context
of the Décysif project.

TODO: Overview of the document: Section~\ref{sec:why3}, Section~\ref{sec:spark},
Section~\ref{sec:j3}, Section~\ref{sec:creusot}, Section~\ref{sec:related},
Section~\ref{sec:future}.

\section{The Memory Model of Why3}
\label{sec:why3}

% \todo[inline]{Premier jet par Claude $\to$ relecture souhaitée par Andrei et
%   Jean-Christophe} % FAIT

The Why3
environment~\cite{filliatre13esop,bobot14sttt,blazy19fmtea,paskevich20isola} is
designed and developed by the Inria partner of Décysif. It is not an environment
for a particular mainstream programming language such as C or Java. Instead, it
is designed as a \emph{generic} environment for deductive verification. This
genericity comes into different aspects. First, Why3 proposes its own
programming language, called WhyML, dedicated to deductive verification, in the
sense that it is designed to ease the generation of verification conditions,
making the proof of these formulas as automatic as possible. This relies on a
second generic aspect of Why3: its ability to communicate with a large set of
provers to discharge the verification conditions~\cite{boogie11why3}. A third
generic aspect of Why3 is its suitability to be used as an intermediate language
for deductive verification: as a matter of fact, WhyML is used as an
intermediate language by the other tools mentioned below: SPARK, J3 and Creusot.

In the WhyML language, mutability of memory is supported, but to control the
difficulties coming from aliasing and separation, mutation of memory is
controlled by a dedicated powerful \emph{static} typing system. This typing
system is expressed on top of important notions of \emph{ownership} and
\emph{regions}~\cite{gondelman16reg}. Thanks to this typing system, the
potential issues with aliasing and memory separation are handled \emph{during}
the generation of VCs, so that the generated formulas do not need to take
aliasing anymore into account. Roughly speaking, the generated VCs are as simple
as if they were coming from a purely functional programming language, with no
memory mutability at all. One must understand in particular that this approach
differs fundamentally with other approaches well-known in the literature, such
as \emph{separation logic}~\cite{Tuch_KN_07} or \emph{dynamic
  frames}~\cite{Smans09}, where non-aliasing and separation properties are still
present in the logic formulas to be proved valid, which makes the work of
provers significantly more difficult. The powerful region-based mechanism of
Why3 comes at a cost though: the set of programs that one is writing in WhyML
must pass the static type checking, which means the programmer must think carefully
about how the side effects in their program must be expressed.

The research work around Why3 is still active nowadays, e.g. regarding
the design of better constructs for writing specifications, the
improvement of the ratio of proof automation, or the generation of
counterexamples when proofs
fail~\cite{dailler18jlamp,becker21fide}. In the context of Décysif,
the efforts are planned on two directions: the design of an
alternative to WhyML, and the improvement of counterexamples
generation to better suit the need of front-ends of the partners. These
are details in Section~\ref{sec:future} below.

\section{The Memory Model of SPARK}
\label{sec:spark}

\todo[inline]{Premier jet par Claude $\to$ rédaction à reprendre par AdaCore,
  Claire (qui peut déléguer!)}

During the 1990s, SPARK emerged as a tool used to formally verify a subset of 
the Ada language~\cite{carre90}. In 2014, AdaCore released
SPARK2014~\cite{mccormick15} as a complete
re-implementation of SPARK, made compatible with Ada 2012, a version of Ada that
introduced specification contracts as part of Ada. Since that time, SPARK has been
using WhyML as an intermediate language, and Why3 as a verification condition
generator.

The Ada subset supported by SPARK highly restricts aliasing. It enforces a policy
to ensure that a mutable object can only be accessed through a single name.
This restriction allows translating SPARK programs into WhyML with a simple
\emph{flat} memory model, where all mutable locations are identified as WhyML
mutable variables. Such a choice makes the generation of VCs simple enough to
obtain a satisfactory level of automation of proofs, and also allows generating
meaningful counterexamples~\cite{dailler18jlamp}.

Because they easily introduce aliasing, pointers (called access types in Ada)
and memory allocation have been disallowed for a long time. In 2020, the
SPARK tool was extended to support them. To control aliasing,
a technique inspired by the Rust borrowing policy and the Prusti tool was
designed and implemented~\cite{dross20cav,jaloyan20icfem}. An essential idea
behind that technique is that each alias is statically known by the type
system so that programs can continue to be translated into an aliasing-free
WhyML program. Later on, the method implemented was slightly modified to follow
the idea of \emph{prophetic encoding} introduced originally by Matsushita \emph{et
al}~\cite{matsushita20esop}.

Nowadays, the major issues with SPARK remain the need to increase the ratio of
automation of proofs, and to produce better counterexamples. The objectives for
SPARK in the Décysif project are detailed in Section~\ref{sec:future} below.

\section{The Memory Model of J3}
\label{sec:j3}

\todo[inline]{Premier jet par Claude $\to$ rédaction à reprendre par TrustInSoft,
  Guillaume (qui peut déléguer!)}

Since its creation in 2013, TrustInSoft develops TIS-Analyzer, an environment
for formal analysis of C and C++ source code. The central method for analyzing
code is an abstract interpretation engine, that is aimed at detecting undefined
behaviors in any kind of C programs. To complement the use of abstract
interpretation when alarms are signaled, it was decided in 2020 to design and implement
a new plug-in, called J3, to apply deductive verification techniques on such C
and C++ code.

The initial design of J3 included important decisions regarding the supported
subset of C. In essence, it was decided that any kind of code should be
supported. That includes low-level C code, making use of complex, low-level,
architecture-dependent, bit-wise manipulation, but also arbitrary use of
pointers, and conversion (cast) of such pointers. Memory allocation and freeing
must be supported too, so as any kind of aliasing and memory overlap. To reach
such an ambiguous goal, but not starting everything from scratch, it was decided
to reuse some existing memory model, namely it was decided to use CompCert
V2~\cite{leroy12rr} memory model, a model already formalized in Coq and the basis of
the C semantics on which the verified C compiler CompCert operates.

The formalization of the memory model for J3 is done using WhyML, it makes heavy
use of bit-vectors~\cite{fumex16nfm}. It explicitly defines the notions of memory
blocks and permissions, so that at the end, generated verification conditions
for a C code contains explicit pointers, predicates expressing separation (or
not separation) and other memory-related concepts. A consequence of this design
is that in its current state, the proofs done using J3 suffer from a poor ratio
of automation. It is one of the objective in Décysif to increase this ratio.

Another important design choice for J3 was the will to obtain a counterexample
when a proof fails. In its current state, J3 is able to produce such
counterexamples in some cases, but overall the methodology needs significant
improvement. Even more, it is a central goal in Décysif to be able to turn such
a counterexample into an \emph{exploit}, that is an explicit test case that
exposes a vulnerability regarding safety or security.

The Décysif objectives for J3 are detailed below in Section~\ref{sec:future}.

\paragraph{Positioning}

J3 is far from being the only tool for deductive verification of C
code. Historically, Frama-C/WP~\cite{blanchard2024wp} is a deductive
verification plugin for Frama-C, the environment from which TIS-Analyzer was
derived. WP design was guided by different choices than those of J3, in
particular it does not attempt to support any kind of memory aliasing. On the
contrary, it provides a set of variants of memory models where separation is
partially or even totally assumed. Also, it wasn't designed with the need of
counterexamples in mind, and only very recently such a feature has been added as
experimental. Comparison between J3 and WP on that matter will be interesting to
investigate in the future.

Jessie is another historic deductive verification plugin for Frama-C, designed
on top of the predecessor of Why3. It is not maintained anymore. J3 design
inherits in part from Jessie. It is in particular planned to resurrect inside J3 a
static separation analysis similar to the one of
Jessie~\cite{hubert2008these,hubert07hav}

VCC~\cite{dahlweid09icse} was historically another tool for deductive
verification of C, inspired by Spec\# ownership (see Section~\ref{sec:related}
below). Not any kind of memory aliasing was supported. It is not maintained
anymore.
%
VST~\cite{appel11esop} is an environment built on top of CompCert and a Coq
formalization of separation logic. To perform a proof of some C code, it is
mandatory to use the Coq system, thus having a low level of automation. An
advantage though is that the underlying memory model is low-level and precise.
%
\mbox{VeriFast}~\cite{jacobs11nfm} is a tool for deductive verification of C (and Java)
code. It relies on separation logic and/or dynamic frames. To perform proofs,
the code must be annotated by the user to explicit the reasoning regarding
opening or closing definitions of memory-related representation predicates.
%
Isabelle/C~\cite{tuong19fide} is an environment for reasoning on C code on top
of Isabelle/HOL and its layer AutoCorres for imperative programs. To perform
proofs it is mandatory to use the Isabelle system interactively.
%
Generally speaking, the tools above are not designed to produce exploits in case of
proof failure, or even counterexamples.

Notice finally that J3 has the ambition to be applied on C++ code, which is
hardly supported by any other tool.

\section{The Memory Model of Creusot}
\label{sec:creusot}

\todo[inline]{Premier jet par Claude, rédaction à reprendre par l'équipe
  Creusot, Jacques-Henri ?}

Creusot is a tool for deductive verification of Rust programs, developed by
Inria. The Rust programming language has a unique feature: it is equipped with a
very elaborated static typing system, so that when a program passes this type
checker, it can be considered \emph{safe} in the sense that it is free from
undefined behavior related to memory access.

Creusot leverages on this property to produce verification conditions for safe
Rust programs, that are do not need to express anymore any property regarding
aliasing and memory separation. To achieve this Creusot uses a memory model
involving a so-called prophetic encoding of mutable borrows, which roughly
amounts to represent a borrow by a pair of its current value and its future
value and the end of its lifetime~\cite{denis22icfem,denis23phd}.  Technically,
Creusot translates Rust into Why3 where memory read and write are encoded in
WhyML mutables variables. Thanks to that approach, the obtained verification
conditions are significantly easier to be discharged by automated provers.  The
soundness of the translation based on prophetic encoding has been proved valid
using Coq and its RustBelt library~\cite{matsushita22pldi}.

\paragraph{Positioning}

Historically, Prusti~\cite{astrauskas19oopsla} is the first tool to implement a
verifier for Rust programs, leveraging on the borrow checker. A significant
difference with Creusot is that Prusti encodes Rust code into Viper, which is an
intermediate language based on separation logic. Technically, this means that
Prusti is not limited to safe Rust as Creusot is, so supports a larger subset of
Rust. On the other hand, the generated verification are significantly harder to
discharge by automated provers, as shown by the benchmarks we
did~\cite{denis23phd}.

TODO positioning with Aeneas~\cite{ho22icfp} \todo{Jacques-Henri}

TODO positioning with AWS \todo{Jacques-Henri}


\section{Additional Comparison and Positioning}
\label{sec:related}

\todo[inline]{Premier jet par Claude, relectures souhaitées}

There are other mainstream programming languages for which deductive
verification tools exists. Historically, even before 2000, Java was one the
first of these languages, starting with the tool ESC-Java~\cite{escjava98} for
which the specification language JML~\cite{Burdy04} was invented. At that time
there were very few techniques for handling conveniently the heap memory. In the
early 2000s, JML featured a notion of \emph{data groups}~\cite{breunesse03ftfjp}
for such a purpose, coming with early notions of \emph{ownership}, as it was
implemented in the Spec\#~\cite{BarnettLS04} tool for C\#, or similar notions of
\emph{universes}~\cite{Dietl05jot}.

Separation Logic~\cite{reynolds02lics} appeared around the same time in 2002,
but at first there were hardly any automated tool using it, because of the lack
of automated reasoning techniques dedicated to this special from of logic.

In 2006 was proposed the concept of \emph{dynamic frames}~\cite{kassios2006fm}
which was seen as an alternative to separation logic more amenable to automated
reasoning, in particular with the variant of \emph{implicit dynamic
  frames}~\cite{Smans09} which was implemented in new tools like
VeriFast~\cite{jacobs11nfm} (2009, for Java and C) and
Dafny~\cite{Dafny,leino14fide} (2010, featuring its own object-oriented
programming language).

Separation Logic obtained a more obvious success in the context of interactive
theorem proving tools. It is exemplified say by the Iris~\cite{iris17} library
for Coq. Separation Logic implemented in Coq gave birth to tools like
Ynot~\cite{nanevski08icfp} and CFML~\cite{chargueraud08icfp}, both dealing with
their own ML-like language. As already mentioned above in Section~\ref{sec:j3},
this line of approach is applied to the C programming language nowadays: the
Verified Software Toolchain (VST) builds upon the CompCert memory model and
separation logic to allow (interactive) proofs of C programs in Coq, and
Isabelle/simple provide an infrastructure for reasoning about imperative
programs in Isabelle/HOL, on top of which is built the
Isabelle/C~\cite{tuong19fide} environment and the
AutoCorres~\cite{greenaway14pldi} tool.

More recently in 2016 appeared the environment
Viper~\cite{MuellerSchwerhoffSummers16}, which is meant as a generic environment
for automated verification using separation logic/dynamic frames. It is used as
an intermediate language for the Prusti tool for Rust (already mentioned in
Section~\ref{sec:creusot}) and Vercors~\cite{armborst24cav}.

Tools exists for the deductive verification of OCaml programs, namely the
Gospel~\cite{gospelfm19} environment which combines different techniques
including separation logic in Coq, and Cameleer~\cite{DBLP:conf/cav/PereiraR20}
which proceeds by translation into Why3. For the Scala programming language
there exist the Stainless tool~\cite{chassot24ijcar}.

To conclude with this section of overall positioning, let's emphasize that with
the tool we develop in Décysif (in Why3, SPARK and Creusot but with the notable
exception of J3) we adopt a general approach were aliasing and separation is
handled in an early stage of static typing based on ownership-like concepts,
allowing to produce verification conditions that do not include anymore
memory-related notions. This make the generated verification conditions more
amenable to automation. J3 is an exception, since there is a will to support all
C features even to most low-level ones. This is why improving the memory model
of J3 remains an important future work in Décysif.

\section{Planned Future Work within Décysif}
\label{sec:future}

We conclude this document with a review of the research actions that are planned
with Décysif related to the memory models and the tools support.

\begin{description}
\item[Flexible Abstraction Barriers]

  \todo[inline]{Premier jet Claude, please amend JCF and Andrei. Andrei: possibly copy the ``perspectives'' document}
  The on-going PhD thesis of Paul Patault is aimed at the design of an
  intermediate language in Why3, as an alternative of WhyML. The goal is to
  relax the standard approach where function calls act as a mandatory abstraction
  barrier: classically, when reasoning about a function call, only the contract
  of that function is visible. The new intermediate language under design, now
  called Coma, is based on a continuation passing style. It is intended to use
  Coma as a target from Creusot, as Coma offers new possibilities in particular
  regarding the translation of Rust closures.

\item[Memory model for Coma]

  In the context of the Coma verification language (see above),
  we want to explore a new abstraction/refinement mechanism for
  programs with mutable state.
  %
  On the implementation side, this mechanism would admit arbitrary
  pointer aliasing, allowing us to verify algorithms with complex
  pointer structure.
  % for which usual ownership-based methods would be too restrictive.
  This verification would rely on memory models, which grant us
  greater expressiveness, at the price of more complex specification
  and reasoning.
  %
  On the interface side, these memory models would be hidden and a
  simple alias-free specification would allow us to apply simpler
  Hoare-style reasoning to verify the client code.
  %
  A typical example would be the design of a reusable software component for
  mutable maps, whose implementation uses hash tables and mutable singly-linked
  lists for buckets, but whose specification only requires a single mutable
  variable containing a dictionary.
  %
  This work is part the ongoing PhD thesis of Paul Patault at Inria.

\item[Future Work in SPARK]

  \todo[inline]{Rédacteur Souhaité: AdaCore, Claire ?}
  
  The main pain points for users of pointers in SPARK currently are the multiple
  restrictions enforced to ensure statically recognizable aliases. We plan to work on
  investigating which restrictions can be relaxed to improve the expressivity of
  the language without losing the simple and efficient encoding of the memory
  model inside WhyML.

  Increasing the level of automation inside the tool is also an axis of improvement.
  In particular, pointer-based recursive data-structures are currently encoded in a
  way which does not always give statisfactory results. We plan to work on improving
  our translation to WhyML and its underlying solvers.

  Finally, we would like to investigate relaxing some of the non-aliasing constraints
  imposed by SPARK in non-executable ghost code. Indeed, such constraints are
  generally necessary to ensure compatibility between the executable semantics
  of the Ada code and how it is encoded for verification inside Why3. On code purely
  meant for analysis, it might be possible to relax some of these constraints without
  compromising soundness, like is done in Creusot.

\item[J3]

  \todo[inline]{Premier jet Claude, please amend Guillaume}

  So far, the memory model of J3 does not give satisfactory results regarding
  the ratio of automated proofs (see benchmarks from deliverable 2.1), and
  there are plans to make it better. One important axis is to implement a static
  analysis of separation, similar to the one of the former plugin
  Jessie~\cite{hubert2008these,hubert07hav}, so that the generated verification
  conditions become simpler, with less need for memory-related predicate.

  The main axis of work on J3 is towards the generation of exploits. It means
  that the potential counterexamples that are obtained from Why3 when proof
  fails (see benchmarks from deliverable 1.1) must be turn into concrete test
  cases for the original C code.

\item[Extensions of Creusot]

  \todo[inline]{Premier jet Claude, please amend Jacques-Henri}

  The on-going PhD thesis of Arnaud Golfouse is aimed at extending the subset of
  Rust that is supported by Creusot. This include support for concurrent
  programs, support for the so-called ghost aliasing.

  Supporting this extended set of Rust features need to reconsider the
  prophecy-based modeling, providing adequate extensions, show them correct,
  implement them, and evaluate them on concrete Rust code.

\item[From Counterexamples to Exploits]

  \todo[inline]{Premier jet Claude, please amend Guillaume}

  As already said above for J3, one major goal in the Décysif project is the
  generation of exploits from alarms obtained in the analyzers. When such an
  alarm is raised at a given location of a source, it means that the analyzer
  was not able to guarantee that the corresponding piece of code is safe
  regarding safety and security. Constructing an exploit is a way to be able to
  decide if such an alarm is a real bug or a false alarm.

  The current objective is to start from potential counterexamples suggested by
  provers (see benchmarks in deliverable 1.1) and turn them into real test cases
  to be replayed with a concrete execution of the suspicious code.

  This objective is shared between J3 and SPARK

\item[Counterexamples in Creusot]

  \todo[inline]{Premier jet Claude, please comment}

  So far Creusot is not designed with the ability to produce counterexamples in
  case of proof failure. This is thus another plan within Décysif to make this
  possible, including the fact that, in a second step, these counterexamples
  should be turned into exploits to be executed on the Rust original code.

\end{description}

\bibliographystyle{plainurl}
%\bibliography{abbrevs,demons,demons2,demons3,team,crossrefs}
\bibliography{generated,extra}

%\clearpage


\end{document}

% Local Variables:
% mode: latex
% TeX-master: t
% TeX-PDF-mode: t
% eval: (flyspell-mode 1)
% ispell-local-dictionary: "american"
% fill-column: 80
% End:

% LocalWords:  Décysif genericity
