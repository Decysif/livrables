\documentclass[a4paper,twoside]{article}

% fonts
\usepackage[utf8]{inputenc}
\usepackage[T1]{fontenc}
\usepackage{mathptmx}
\usepackage[scaled=0.8]{beramono}

% geometry
\usepackage[a4paper,centering]{geometry}
\setlength{\marginparwidth}{2cm}

% Inria research report
\usepackage{RR}

% extra packages

\usepackage{hyperref}
\usepackage{xspace}

\usepackage{amsmath}
\usepackage{amsfonts}
\usepackage{amsthm}

\usepackage[color=yellow!45!white,linecolor=black,textsize=footnotesize]{todonotes}

\usepackage{why3lang}

% placement of figures
\renewcommand\floatpagefraction{0.99}
\renewcommand{\textfraction}{0.01}
\renewcommand{\topfraction}{0.99}
\renewcommand{\bottomfraction}{0.99}
\setcounter{topnumber}{4}
% \setcounter{bottomnumber}{4}
% \setcounter{totalnumber}{8}


\newcommand{\ie}{\textit{i.e.\/}\xspace}
\newcommand{\eg}{\textit{e.g.\/}\xspace}


\RRNo{XXXX}    % TODO
% \RTNo{XXXX}
%%
%% date de publication du rapport
\RRdate{xxx 2024}
%%
%% Cas d'une version deux
%% \RRversion{2}
%% date de publication de la version 2
%% \RRdater{November 2008}
%%
\RRauthor{% les auteurs
  Claude Marché\thanks[toccata]{Université Paris-Saclay, CNRS, ENS Paris-Saclay, Inria,
    LMF, 91190, Gif-sur-Yvette, France}
}
%% Ceci apparait sur chaque page paire.
\authorhead{Marché}
%% titre francais long
\RRtitle{Un état de l'art des modèles mémoires pour les programmes à pointeurs\thanks{Ces recherches ont été partiellement financées par Décysif TODO....}}
%% English title
\RRetitle{A state-of-the-art of memory models for pointer programs\thanks{This work has been partially supported by Décysif... TODO}}
%%
\titlehead{Memory Models for Pointer Programs}
%%
% \RRnote{This is a note}
% \RRnote{This is a second note}
%%
\RRresume{A FAIRE}

\RRabstract{TO DO}

%%
\RRmotcle{Vérification déductive, Programmes à pointeurs, Modèles mémoire}
%
\RRkeyword{Deductive Verification, Pointer programs, Memory models}
%%
\RRprojet{Toccata}  % cas d'un seul projet
% \RRprojets{Apics and Op\'era and Marelle}
%%
%% \URLorraine % pour ceux qui sont \`a l'est
%% \URRennes  % pour ceux qui sont \`a l'ouest
%% \URRhoneAlpes % pour ceux qui sont dans les montagnes
%% \URRocq % pour ceux qui sont au centre de la France
%% \URFuturs % pour ceux qui sont dans le virtuel
%% \URSophia % pour ceux qui sont au Sud.
%%
%% \RCBordeaux % centre de recherche Bordeaux - Sud Ouest
%% \RCLille % centre de recherche Lille Nord Europe
%% \RCParis % Paris Rocquencourt
\RCSaclay % Saclay \^Ile de France
%% \RCGrenoble % Grenoble - Rh\^one-Alpes
%% \RCNancy % Nancy - Grand Est
%% \RCRennes % Rennes - Bretagne Atlantique
%% \RCSophia % Sophia Antipolis M\'editerran\'ee


% \maketitle

\newtheorem{definition}{Definition}[section]
\newtheorem{example}[definition]{Example}
\newtheorem{lemma}[definition]{Lemma}
\newtheorem{corollary}[definition]{Corollary}
\newtheorem{theorem}[definition]{Theorem}


\begin{document}
\sloppy\hbadness=9999
%%
\makeRR   % cas d'un rapport de recherche
% \makeRT % cas d'un rapport technique.
%% a partir d'ici, chacun fait comme il le souhaite


\tableofcontents
\clearpage

\listoffigures

\clearpage

\section{Introduction}

What is deductive verification, what is the challenge with pointer programs, aliasing,
what are memory models.

The case of Décysif project: the case of tools Why3, SPARK, J3 and Creusot.

Generalities about memory models

Overview of the document.

\section{The memory model of Why3}

The region-based aliasing analysis of Why3~\cite{gondelman16reg}.

Common current issues. E.g. with specification, with proof automation, with
counterexamples.

\section{The memory model of Spark}

Complete absence of pointers at first, then partial support of pointers, aliased
controlled using a borrow checker~\cite{dross20cav,jaloyan20icfem}.

How the translation to Why3 works.

Common current issues. E.g. with specification, with proof automation, with
counterexamples.

\section{The memory model of J3}

Aims at supporting low-level C code, with arbitrary pointer casts. Inspired by
CompCert V2~\cite{leroy12rr}. Not yet supported but future work: static analysis
of separation similar to the old plugin
Jessie~\cite{hubert2008these,hubert07hav}.

How the translation to Why3 works.

\section{The memory model of Creusot}

Relies on Rust's borrow checker to propose a translation to Why3. mutable
borrows are translated to pairs of current value and future final value
(prophecy).\cite{denis23phd,denis22icfem,matsushita22pldi}.

\section{Conclusions and Future Work}

Future work
\begin{itemize}
\item New intermediate language in Why3, continuation-based. On going thesis of
  Paul Patault.
\item Separation analysis for J3.
\item Support for concurrency in Creusot, Support for ghost aliasing, on-going
  thesis of Arnaud Golfouse.
\item Better support for counterexamples, needs to makes memory models more
  concrete, more ``executable-like''
\end{itemize}

\bibliographystyle{plainurl}
%\bibliography{abbrevs,demons,demons2,demons3,team,crossrefs}
\bibliography{generated,extra}

%\clearpage


\end{document}

% Local Variables:
% mode: latex
% TeX-master: t
% TeX-PDF-mode: t
% mode: flyspell
% ispell-local-dictionary: "british"
% fill-column: 80
% End:
