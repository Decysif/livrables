\documentclass[a4paper,11pt]{article}

\usepackage[utf8]{inputenc}
\usepackage[T1]{fontenc}
\usepackage{mathptmx}

\usepackage[a4paper,text={160mm,220mm},centering]{geometry}

\usepackage{graphicx}



\begin{document}


\unitlength=1mm
\begin{picture}(0,0)(0,10)
\put(-20,25){\includegraphics[width=0.3\textwidth]{../images/Logo_Bpifrance.png}
  \includegraphics[width=0.1\textwidth]{../images/Logo-France-2030-rouge-bleu.png}}
\end{picture}

\begin{center}\bfseries

  \Huge
  Projet Décysif --- Livrable 1.1

  \Large
  Constitution d’une base de fichiers d’entrée
représentatifs des difficultés rencontrées pour
générer des exploits.
\end{center}


Ce livrable est constituée d'une base de tests qui se trouve
dans le dépot 'livrables' du projet Decysif (TODO: URL).

Objectif du livrable:
- Repérer les faiblesses d'alt ergo
- Repérer les problèmes de traduction (ou repérer des problèmes au niveau de l'écriture des théories, par exemple le modèle mémoire de J3) pour tous les prouveurs cvc5, CVC4, Z3, Alt-Ergo.

% Une section par répertoire d'exemples avec une description du contenu
% et de la méthodologie des statisques adoptée. Et le résultat de ces statisques
% au démarrage du projet.

% Mettre les statistiques qu'on a quand elles existent.

\section{Examples issus de Why3}

TODO [Claude]

\section{Examples issus de J3}

TODO [Guillaume]

Méthodologie pour extraire des statistique à décrire:
* Nombre de buts prouvés/Nombre de buts au total
TODO: Fixer la version d'alt-ergo

\section{Examples issus de SPARK}

TODO [Yannick]

\section{Examples issus de Creusot}

TODO [Claude]

\section{Travail futur}

Quelles autres statistiques peut-on vouloir?

\end{document}